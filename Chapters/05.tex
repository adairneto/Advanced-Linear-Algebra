\chapter{Elementary Canonical Forms}

\section{Introduction}

The question that motivates this section is `when the matrix of a linear operator assumes a simple form?'

Consider, for example, the following diagonal matrix.
\[
D = \begin{bmatrix}
c_1 & 0 & 0 & \ldots & 0 \\
0 & c_2 & 0 & \ldots & 0 \\
0 & 0 & c_3 & \ldots & 0 \\
\vdots & \vdots & \vdots & \ddots & \vdots \\
0 & 0 & 0 & \ldots & c_n
\end{bmatrix}
\]

And suppose that $T$ is a linear operator on a finite vector space $V$. If there exists an ordered basis $\beta = \{ v_1, v_2, \ldots, v_n \}$ of $V$ in which $T$ is represented as the diagonal matrix $D$, then it is possible to extract some informations about the linear operator $T$, such as its rank and determinant, in a simple and direct way.

Since 
\[
[T]_\beta = D \iff T(v_k) = c_k v_k, k = 1, 2, \ldots, n
\]
the range of $T$ is simply the subspace spanned by the vectors $v_k$ in which $c_k$ does not vanish. Analogously, the null space  of $T$ is generated by the remaining $v_k$'s. 

Is it always possible to represent a linear operator $T$ as a diagonal matrix? If not, what is the simplest type of matrix by which we can represent $T$?

%\begin{equation*}
%	\begin{aligned}
%	
%	\end{aligned}
%\end{equation*}