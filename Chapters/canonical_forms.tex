\chapter{Canonical Forms}

Our goal in this chapter is to break a vector space into `important' subspaces with respect to a polynomial associated with a given linear operator.

Here we'll use $V$ to denote a vector space over $\mathbb{F}$, $T \in \text{End}(V)$, and $V_p := \{ v \in V : p(T)(v) = 0 \}$, i.e., the nullspace of the operator, where $p \in \mathbb{F}[x]$.

\section{Annihilating Polynomials}

Recalling that $T^0 := \{ p \in \mathbb{F}[x] : p(T) = 0 \}$ is the annihilator of $T$, it follows that the collection of polynomials $p$ which annihilate $T$ is an \hyperref[def:ideal]{ideal} in the polynomial algebra $\mathbb{F}[x]$.

We'll state the following theorem and proceed with a discussion before proving each of its parts.

\begin{theorem}[Existence of the Minimal Polynomial]\label{thm:minimal-polynomial} \hfill
	\begin{enumerate}
		\item If $\dim (V)$ is finite, then $T^0 \neq \{ 0 \}$.
		\item If $T^0 \neq \{ 0 \}$, then there exists a unique monic polynomial $m_T \in \mathbb{F}[x]$ such that $m_T$ divides every element of $T^0$.
	\end{enumerate}
\end{theorem}

\begin{definition}[Minimal Polynomial]
	The polynomial $m_T$ in the previous theorem is called \textbf{minimal polynomial} of $T$.
	
	If $T^0 = \{ 0 \}$, we define $m_T = 0$.
\end{definition}

Since every polynomial ideal consists of all multiples of some fixed monic polynomial (the generator of the ideal), we may define the minimal polynomial for $T$ as the unique monic generator of the ideal of polynomials over $\mathbb{F}$ which annihilate $T$.

Summarizing, the minimal polynomial $p$ for the linear operator $T$ is uniquely determined by these three properties:
\begin{enumerate}
	\item $p$ is a monic polynomial over $\mathbb{F}$;
	\item $p(T) = 0$;
	\item No polynomial over $\mathbb{F}$ which annihilates $T$ has smaller degree than $p$ has.
\end{enumerate}

These definitions can be easily extended to a matrix $A$ instead of an operator $T$. Moreover, it follows from previous remarks that similar matrices have the same minimal polynomial.

The next example shows that it is possible to have $T^0 = \{ 0 \}$ when the dimension is infinite.

\begin{example}
	If $V = \mathbb{F}[x]$ and $T$ is the operator $f(t) \mapsto tf(t)$, then $T^0 = \{ 0 \}$.
\end{example}

\begin{theorem}[Primary Decomposition Theorem (PDT)]
	Suppose that $T^0 \neq \{ 0 \}$ and let $p_1, \ldots, p_m$ be the distinct irreducible factors of $m_T$ in $\mathbb{F}[x]$. If $k_j$ is the multiplicity of $p_j$ in $m_T$, then
	\[
		V = V_{p_1^{k_1}} \oplus \ldots \oplus V_{p_m^{k_m}}
	\]
	
	The polynomials $p_j^{k_j}$ are called the \textbf{primary factors (pf)} of $T$ and $V_{p_j^{k_j}}$ is said to be a \textbf{$T$-primary subspace} of $V$.
\end{theorem}

In words, if there is non-zero polynomial in the annihilator, then there exists a minimal polynomial. Considering the prime factors of the minimal polynomial and its multiplicities, we can `break' the vector space as direct sum of certain subspaces, where each subspace is related to one of the factors of the minimal polynomial to its multiplicity.

It follows immediately from the PDT that a linear operator is diagonalizable iff. its primary factors have degree one.

Recall that a subspace $W$ of $V$ is said to be $T$-invariant if $T(W) \subseteq W$. In particular, the restriction of $T$ to $W$ induces a linear operator in $W$ given by $w \mapsto T(w)$ for all $w \in W$.

\begin{lemma}
	If $S \in \text{End}(V)$ satisfies $S \circ T = T \circ S$, then the nullspace of $S$, $V_S$, is $T$-invariant. In particular, $V_p$ is $T$-invariant for all $p \in \mathbb{F}[x]$.
\end{lemma}

\begin{proof}
	Let $v \in V_S$ and notice that $S(T(v)) = T(S(v)) = 0$, which shows that $T(v) \in V_S$.
\end{proof}

Generalizing the notion of eigenspace. To do that, we want to describe the following set
\begin{equation}\label{eq:set-goal-describe}
	\{ p \in \mathbb{F}[x] : V_p \neq \{ 0 \} \}
\end{equation}

We already know that the minimal polynomial is inside this set.

For any two polynomials $f, p$, then $V_p \subseteq V_{fp}$. In particular, $V_p^k \subseteq V_p^{k+1}$ for all $k \geq 0$. Therefore, the set
\[
	V_p^\infty := \bigcup_{k > 0} V_p^k
\]
is a $T$-invariant subspace of $V$.

\begin{definition}[Generalized Eigenspace]
	For $p(t) = t - \lambda$, where $\lambda \in \mathbb{F}$, the space $V_p^\infty$ is called the \textbf{generalized eigenspace} associated with $\lambda$.
\end{definition}

\begin{example}
	If $T \in \text{End}(\mathbb{F}^2)$ is given by $T(x,y) = (y,0)$ and $p(t) = t$, then
	\[
		V_p = V_T = [e_1] \quad \text{ and } \quad V_p^2 = \mathbb{F}^2 ~\text{ (since $T^2 = 0$)}
	\]
	
	Is there any other polynomial such that $V_p \neq \{ 0 \}$?
	
	If $f \in \mathbb{F}[x]$ and $p \nmid f$, then $V_f = \{ 0 \}$. In fact, if $f = pq+r$ is the division of $f$ by $p$, then $r \in \mathbb{F} \setminus \{ 0 \}$ (since $p$ has degree one and $p \nmid f$) and
	\[
		f(T)(x,y) = q(T)(T(x,y)) + r(x,y) = q(T)(y,0) + r(x,y) = (q(0)y + rx, ry)
	\]
	
	Therefore, if $r \neq 0$, 
	\[
		F(T)(x,y) = (0, 0) \iff 0 = y = x
	\]
\end{example}

The following proof, for the second item of the Theorem \ref{thm:minimal-polynomial}, shows the existence of minimal polynomial and how to find it.

\begin{theorem}
	If $T^0 \neq \{ 0 \}$, then there exists a unique monic polynomial $m_T \in \mathbb{F}[x]$ such that $m_T$ divides every element of $T^0$.
\end{theorem}

\begin{proof}
	Let $m = \min \{ k : \exists p \in T^0 \setminus \{ 0 \}, \deg(p) = k \} > 0$, i.e., the smallest degree of a non-zero polynomial in the annihilator.
	
	Fix $f, p \in T^0$ with $\deg(p) = m$. By Euclid's Division Algorithm, $f = qp + r$, where $\deg(r) < m$. Notice that $r = f - qp \in T^0$. By the minimality of $m$, $r = 0$ and therefore $p | f$.
	
	This shows that any non-constant polynomial with the minimal degree divides every other polynomial in the annihilator. And a polynomial divides another polynomial with the same degree if one is a scalar multiple of the other.
\end{proof} 

To describe the set \eqref{eq:set-goal-describe}, it is sufficient to consider the following lemma and proposition.

\begin{lemma}
	If $\gcd(f,g) = 1$, then the restriction of $f(T)$ to $V_g$ is injective.
\end{lemma}

\begin{proof}
	Let $p, q \in \mathbb{F}[x]$ such that $pf + qg = 1$. Then, for every $v \in V_g$,
	\[
		v = (p(T)f(T) + q(T) g(T))(v) = (p(T)f(T))(v)
	\]
	since $g(T)(v) = 0$. It follows that the restriction of $p(T) \circ f(T)$ to $V_g$ is the identity function. Hence, the lemma follows.
\end{proof}

\begin{proposition}
	If $T^0 \neq \{ 0 \}$ and $p \in \mathbb{F}[x]$ is irreducible, then $V_p \neq \{ 0 \}$ iff. $p | m_T$.
\end{proposition}

\begin{proof}
	Suppose that $p | m_T$ and $V_p = \{ 0 \}$, i.e., $p(T)(u) \neq 0$ for all $u \in V \setminus \{ 0 \}$. Consider $f = m_T / p$ and notice that
	\[
		0 = m_T(T)(v) = p(T)(f(T)(v)) \quad \forall~v \in V
	\]
	
	If $f(T)(v) \neq 0$, then $p(T)(f(T)(v)) \neq 0$. Thus, $f \in T^0$, which is a contradiction since $\deg(f) < \deg(m_T)$.
	
	Reciprocally, if $p \nmid m_T$, then $\gcd(p, m_T) = 1$, since $p$ is irreducible. If follows from the lemma that the restriction of $m_T(T)$ to $V_p$ is injective. Since $m_T(T) = 0$, we can conclude that $V_p = \{ 0 \}$. 
\end{proof}

Put another way, a prime polynomial lives in the set iff. it divides the minimal polynomial.

In a finite dimensional vector space, how can we compute the minimal polynomial?

Finally, we prove the first item of the Theorem \ref{thm:minimal-polynomial}.

\begin{theorem}
	If $\dim(V) = n < \infty$, then $T^0 \neq \{ 0 \}$.
\end{theorem}

\begin{proof}
	If $T = 0$, then $T^0 = \{ p \in \mathbb{F}[x] : p(0) = 0 \} \neq \{ 0 \}$. In this case, the minimal polynomial is $p(t) = t$.
	
	Suppose that $T \neq 0$ and remember that $\dim(\text{End}(V)) = n^2$. Then we can construct the family $(T^k)_k$, where $k = 0, 1, \ldots, n^2$, which is linearly dependent (since it contains $n^2 + 1$ elements).
	
	Since the subfamily given by $T^0 = I_V$ is linearly independent, there exists $1 \leq m \leq n^2$ minimal such that the subfamily $(T^k)_{k=0, \ldots, m}$ is linearly dependent.
	
	Let $a_0, \ldots, a_{m-1} \in \mathbb{F}$ such that
	\[
		T^m = a_0 I_V + \ldots + a_{m-1} T^{m-1} \text{ and } f(t) = t^m - \sum_{k=0}^{m-1} a_k t^k
	\]
	
	It follows that $f(T) = 0$ and, therefore, $f \in T^0 \setminus \{ 0 \}$.
\end{proof}

\textbf{Exercise.} Prove that $f = m_T$, where $f$ is as in the preceeding proof.

How can we use this fact and matricial representations of $T$ to compute $m_T$?

\section{Cyclic subspaces}

Given $v \in V$, consider the sequence of vectors
\[
	v_0 = v, v_1 = T(v), \ldots, v_k = T^k(v), \ldots
\]
which is called a \textbf{$T$-cycle} generated by $v$. We denote it by $\mathfrak{C}_T^\infty (v) = (v_k)_{k \geq 0}$ and $\mathfrak{C}_T^m (v) = v_0, v_1, \ldots, v_{m-1}$.

Define the \textbf{$T$-cyclic subspace generated by $v$} as
\[
	C_T(v) = [\mathfrak{C}_T^\infty (v)]
\]

If $\dim(V)$ is finite, there exists $m \geq 0$ minimal such that $\mathfrak{C}_T^{m+1}(v)$ is linearly dependent. Notice that $m \geq 1$ if $v \neq 0$. In particular, if $\mathfrak{C}_T(v) := \mathfrak{C}_T^{m}(v)$ is a basis for $C_T(v)$ and $v_m$ is a linear combination of the linearly independent vectors $v_0, \ldots, v_{m-1}$. And
\[
	v_m = a_0 v_0 + \ldots + a_{m-1} v_{m-1} = \sum_{k=0}^{m-1} a_k T^k(v)
\]

Defining $m_{T,v}(t) = t^m - a_{m-1}t^{m-1} - \ldots - a_1t - a_0$, called the \textbf{minimal polynomial of $v$ with respect to $T$}, 
\[
	m_{T, v}(T)(v) = v_m - \sum_{k=0}^{m-1} a_k T^k(v) = 0
\]
which proves that $v \in V_{m_{T,v}}$. Since $V_{m_{T,v}}$ is $T$-invariant, it follows that
\[
	C_T(v) \subseteq V_{m_{T,v}}
\]

Cyclic subspaces are useful to find divisors of $m_T$. Since $T$ is fixed, we'll write $m_v = m_{T,v}$. 

\begin{theorem}
	For all $v \in V$, $m_v | m_T$.
\end{theorem}

\begin{proof}
	Since $V_{m_v}$ is $T$-invariant, consider the induced operator
	\[
		S : V_{m_v} \longrightarrow V_{m_v}, \quad v \mapsto T(v)
	\]
	
	Note that $m_v \in S^0$. Thus, $m_S | m_v$. In fact, $m_v = m_S$, since if $\deg(m_S) = m' < m$, then $v_0, \ldots, v^{m'}$ would be linearly dependent, contradicting the minimality of $m$.
\end{proof}

%\begin{equation*}
%	\begin{aligned}
%	
%	\end{aligned}
%\end{equation*}